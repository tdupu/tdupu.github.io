\documentclass[11pt, letterpaper]{article}
\usepackage[utf8]{inputenc}

\usepackage[margin=1in]{geometry}

\usepackage{amsmath,amsfonts,amssymb,amsthm,pgf,tikz,enumerate,xcolor,graphicx,microtype}
\usepackage[colorlinks]{hyperref}

\theoremstyle{definition}
\newtheorem{defn}{Definition}
\newtheorem{ex}{Example}

\theoremstyle{remark}
\newtheorem*{rmk}{Remark}

\begin{document}

\begin{center}
{\Large {\sc MATH 6441 -- Graduate Complex Analysis}}
\\[1em]
{\large Fall 2024}
\end{center}

\begin{center}
	\rule{6in}{0.4pt}
	\\[2pt]
	\begin{tabular}{llcll}
		{\bf Instructor:} &Taylor Dupuy &\quad &	{\bf Email:}&\href{mailto:taylor.dupuy@gmail.com}{taylor.dupuy@gmail.com} \\
		{\bf Section A:} &MWF 9:40am -- 10:30am & \quad &{\bf Room:}& WATERMAN 423 \\
	\end{tabular}
	\rule{6in}{0.4pt}
\end{center}

\vspace{2em}

\noindent {\bf Office hours:} Innovation 439,TBD
\vspace{1em}

\noindent {\bf Course webpage:} \url{https://tdupu.github.io}
\vspace{1em}

\noindent {\bf Description and objectives:} 
This is a graduate course in complex analysis. 
Emphasis will be placed on preparation for MS and PhD qualifying exams. 
Topics include: differentiability of complex functions, complex integration and Cauchy's theorem, series expansions, calculus of residues, conformal mapping, and time permitting, the Prime Number Theorem.
\par 

\vspace{1em}

\noindent {\bf Books:} 
\begin{itemize}
\item (main) \emph{Function Theory of One Complex Variable 2nd Ed} Greene, Krantz

\item (supplement) There are a number of textbooks listed on the course webpage. 
\end{itemize}
\vspace{1em}

\noindent {\bf Technology:} No special technology is needed. 
No blackboard. 
No online homework. 
In the rare event that we move online due to an unforeseen COVID surge we will meet on MSTeams.
\vspace{1em}



\noindent {\bf Types of assignments}
Grades will be determined by in class quizzes based on problems from a problem bank. 
\vspace{1em}

\noindent {\bf Important dates:} 
\begin{center} 
	\begin{minipage}{3.8in}
		\begin{flushleft}
			Add/Drop, Pass/No Pass, Audit Deadline \dotfill ~Jan 27 \\
			Last Day to Withdraw \dotfill ~Mar 28 \\
			Last Day of Classes \dotfill ~May 2 \\
		\end{flushleft}
	\end{minipage}
\end{center}
\vspace{1em}



\vspace{1em}

\noindent {\bf Expectations:} Students are expected to regularly attend class, complete any assigned work, and comply with UVM's \href{http://www.uvm.edu/policies/student/studentcode.pdf}{{\em Code of Student Conduct}}.
Moreover, students are expected to act with {\em academic integrity}. That is, the student may not plagiarize or fabricate any work, nor may the student collude or cheat. See UVM's \href{https://www.uvm.edu/policies/student/acadintegrity.pdf}{{\em Code of Academic Integrity}}.
\vspace{1em}

\noindent {\bf ADHD statement} I have combined type ADHD. 
This sometimes manifests itself in typos and oversights in my first time teaching a course (this is my first time teaching this class). 
If you find any mistakes or I forget to respond to something give me a nudge.
I thank you in advance for your patience.
\vspace{1em}

\noindent {\bf Student learning accomodations:} In keeping with University policy, any student with a documented disability interested in utilizing ADA accommodations should contact Student Accessibility Services (SAS), the office of Disability Services on campus for students. SAS works with students and faculty in an interactive process to explore reasonable and appropriate accommodations, which are communicated to faculty in an accommodation letter. All students are strongly recommended to discuss with their faculty the accommodations they plan to use in each course. Faculty who receive Letters of Accommodation with Disability Related Flexible accommodations will need to fill out the Disability Related Flexibility Agreement. Any questions from faculty or students on the agreement should be directed to the SAS specialist who is indicated on the letter. 
\begin{tabular}{ll}
{\em Contact SAS:} &A170 Living/Learning Center; \\
&802-656-7753 \\ 
&\href{mailto:access@uvm.edu}{access@uvm.edu} \\
&\href{https://www.uvm.edu/access}{www.uvm.edu/access}
\end{tabular}

\vspace{1em}

\noindent {\bf Religious holidays} Students have the right to practice the religion of their choice. If you need to miss class to observe a religious holiday, please submit the dates of your absence to me in writing by the end of the second full week of classes. You will be permitted to make up work within a mutually agreed-upon time. See \href{https://www.uvm.edu/registrar/religious-holidays}{www.uvm.edu/registrar/religious-holidays}.
\vspace{1em}

\noindent {\bf FERPA rights disclosure:} The purpose of this policy is to communicate the rights of students regarding access to, and privacy of their student educational records as provided for in the Family Educational Rights and Privacy Act (FERPA) of 1974. See \href{http://catalogue.uvm.edu/undergraduate/academicinfo/ferparightsdisclosure/}{here} for the disclosure.
Sensitive emails should be sent to \href{mailto:tdupuy@uvm.edu}{tdupuy@uvm.edu}.
\vspace{1em}

\noindent {\bf Promoting health and safety:} \\
{\em Center for Health and Wellbeing}:
\url{https://www.uvm.edu/health} \\
{\em Counseling \& Psychiatry Services} (CAPS):
Phone: (802) 656-3340 \\
{\em C.A.R.E.}: If you are concerned about a UVM community member or are concerned about a specific event, we encourage you to contact the Dean of Students Office (802-656-3380).  If you would like to remain anonymous, you can report your concerns online by visiting the Dean of Students website at \url{https://www.uvm.edu/studentaffairs}
\vspace{1em}



 \end{document}