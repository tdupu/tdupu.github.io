\documentclass[a4paper,10pt]{article}
\usepackage[utf8]{inputenc}
\usepackage{amssymb}
\usepackage{amsmath}
\usepackage{fullpage}
\newcommand{\ZZ}{\mathbb{Z}}
\newcommand{\RR}{\mathbb{R}}
\newcommand{\CC}{\mathbb{C}}
\newcommand{\HH}{\mathbb{H}}
\newcommand{\PSL}{\operatorname{PSL}}
\newcommand{\SL}{\operatorname{SL}}
\newcommand{\Stab}{\operatorname{Stab}}
\newcommand{\Arg}{\operatorname{Arg}}
\renewcommand{\Re}{\operatorname{Re}}
\renewcommand{\Im}{\operatorname{Im}}
\newcommand{\hol}{\operatorname{hol}}
\newcommand{\PP}{\mathbb{P}}
\newcommand{\Aut}{\operatorname{Aut}}

%opening
\title{}
\author{Dupuy --- Complex Analysis Problem Bank}
\date{} 


\begin{document}

\maketitle
\tableofcontents

\newpage

\section{Euler's Formula}
Euler's formula states that $e^{i\theta} =\cos(\theta) +i\sin(\theta)$ for $\theta \in \RR$. 
There are some nice things you can do with this. 

\begin{enumerate}
\item Compute and draw the 8th roots of unity.
\item Let $\zeta_n$ be a primitive $n$th root of unity. Show that $\sum_{j=0}^{n-1} \zeta_n^j=0$.

\item (Wallis' Formula) Using the complex representation of cosine, find a formula for 
 $$ \int_0^{2\pi} \cos(\theta)^{2n} d\theta. $$
 
\end{enumerate}

\newpage

\section{Quaternion Exercise}

This exercise show how nice the complex numbers are and how if one tries to develop a notion of holomorphic function in higher for the quaternions.
The quaternions are the Division algebra (noncommutative field) over the reals defined by 
$$\HH= \RR \oplus \RR i \oplus \RR j \oplus \RR k, \mbox{($\cong \RR^4$ as a vector space) } $$ 
where $i$,$j$ and $k$ satisfy
$$ ijk = -1 \mbox{ and } i^2 = j^2 = k^2 = -1.$$
The norm on the quaternions is defined as 
$$ \vert a+bi+cj+dk \vert^2 = a^2 + b^2 + c^3 + d^2,$$
here $a,b,c,d \in\RR$.

\begin{enumerate}
\item 
For $U\subset \HH$ open, we say a function $f: U \to \HH$ is \textbf{holomorphic} if 
$$ f(q) = \lim_{h\to 0} \left( (f(q+h)-f(q))h^{-1}  \right). $$


Show that the only quaternionic holomorphic functions are of the form 
$$f(q) = \alpha q + \beta.$$
where $\alpha,\beta \in \HH$.
\end{enumerate}

\newpage 
\section{Power and Laurent Series}

Many of the problems here were taken from old University of New Mexico qualifying exams. They were automatically converted from a scanned file using Google's Gemini so be wary that these were automatically converted.

\begin{enumerate}

\item Prove Hadamard's formula for the radius of convergence of a series $\sum_{n=0}^{\infty} a_n z^n$.
 $$ R = \lim_{n\to \infty} \inf_{m\geq n} \vert a_m \vert^{-1/m}.$$
Also show that the series converges absolutely and uniformly, (the differentiability thing is the next problem).

\item (Analytic implies Holomorphic) Suppose that $f(z) = \sum_{n\geq0} a_n z^n$ has a radius of convergence $R$. 
\begin{enumerate}
 \item Show $\sum_{n\geq 0} n a_n z^{n-1}$ converges with the same radius of convergence $R$.
 \item Show $\frac{d}{dz}\left[ \sum_{n=0}^{\infty} a_nz^{n}\right] = \sum_{n\geq 0} \frac{d}{dz}[ a_n z^{n}]$ on the disc of convergence.
\end{enumerate}
(Warning: it is not true that for general $u_n(t)\to u(t)$ uniformly that $u_n'(t) \to u_n'(t)$ uniformly! This is a special fact about power series.)


\item Suppose that $f(z) = a_0 + a_1(z-z_0) + a_2 (z-z_0)^2 + \cdots$ has a finite radius of convergence. 
Let $g(z) = a_n+ a_{n+1} (z-z_0) + a_{n+2}(z-z_0)^2 + \cdots$. Show that $g(z)$ has the same radius of convergence as $f(z)$ at $z_0$. (Hint: don't think about this too much)


\item (Extra Credit, see WW page 59) This is a famous example of non-uniform coonvergence. Show that the series
 $$ \sum_{n =1}^{\infty} \frac{z^{n-1} }{(1-z^n)(1-z^{n+1})}$$
 converges to $\frac{1}{(z-1)^2}$ when $\vert z \vert <1$ and $\frac{1}{z(z-1)^2}$ when $\vert z \vert > 1$


\item 
If the series converges do some analysis to determine the radius of convergence at the boundary.
\begin{enumerate}
\item Expand $\frac{1}{1+z^2}$ in a power series around $z=0$, find the radius of convergence.
\item Find the radius of convergence of $\sum_{n\geq 0} n! z^n$.
\item (New Mexico, Jan 1998) Expand $\frac{z^2+2z-4}{z}$ in a power series around $z=1$ and find its radius of convergence.
\end{enumerate}


\item 
\begin{enumerate}
\item 
Let $B\times A \subset \CC \times \CC$ be an open region with compact closure.
Let $f: B\times A \to \CC$ be a function.  Let $\gamma \subset A$ be a $C^1$-curve (so it has finite length). 
Define $F: B \to \CC$ by  
 $$ F(z) = \int_{\gamma} f(z,s) ds.$$
Assuming $\frac{\partial f}{\partial z}(z,s)$ exists and is continuous for all $s \in \gamma$ and all $z \in B$ show that 
 $$ \frac{d}{dz}[ F(z)] = \int_{\gamma} \frac{\partial f}{\partial z}(z,s) ds. $$
\item Let $\Omega \subset \CC$ be an open set. Let $\gamma:[0,1]\to \Omega$ be an $C^1$ curve. 
Let $f \in \hol(\Omega)$ and $g \in L^2(\Omega)$. Show that 
 $$ F(z) := \int_{\gamma} f(\zeta-z)g(\zeta)d \zeta $$ 
is holomorphic on $\Omega$.
\end{enumerate}


\item 
\begin{enumerate}
\item (Whittaker and Watson, page 99)
Consider the series 
 $$ \frac{1}{2} \left( z + \frac{1}{z} \right) + \sum_{n=1}^\infty \left( z - \frac{1}{z} \right) \left( \frac{1}{1+z^n} - \frac{1}{1+z^{n-1}} \right). $$
Show that this series converges for all values of $z$ with $\vert z \vert \neq 1$. 
Furthermore, show that 
$$ \frac{1}{2} \left( z + \frac{1}{z} \right) + \sum_{n=1}^\infty \left( z - \frac{1}{z} \right) \left( \frac{1}{1+z^n} - \frac{1}{1+z^{n-1}} \right) = \begin{cases}
z, & \vert z \vert <1 \\
\frac{1}{z}, & \vert z \vert > 1
\end{cases}$$

\item (Whittaker and Watson, 2.8, problem 16)
By converting the series 
 $$1 + \frac{8q}{1-q} + \frac{16q^2}{1+q^2} + \frac{24q^3}{1-q^3} + \cdots $$ 
(in which $\vert q \vert<1$), into a double series, show that it is equal to 
 $$ 1 + \frac{8q}{(1-q)^2}+ \frac{8q^2}{(1+q^2)^2} + \frac{8q^3}{(1-q^3)^2} + \cdots $$
 \end{enumerate}	
 	
 	\item Consider the following meromorphic functions.
 	
 	\begin{enumerate}
 		\item Expand $f(z) = \frac{1}{2z - z^2}$ in a power series about $z = 1$.
 		
 		\item Find a Laurent expansion of $f(z) = \frac{1}{z} + \frac{1}{z + 2} + \frac{1}{(z - 1)^2}$ which is valid in the annulus $1 < |z| < 2$.
 	\end{enumerate}
 	
 	\item Expand $e^{1/z}$ in a Laurent series about $z = 0$, determining the Laurent coefficients.
 	
 	\item Show that 
 	\[\frac{1}{n!} = \frac{1}{\pi} \int_{0}^{\pi} e^{\cos \theta} \cos(n \theta - \sin \theta) d\theta.\]
 	
 	\item Expand $f(z) = \frac{z + 6}{z^2 - 2z - 3}$ in
 	
 	\begin{enumerate}
 		\item Taylor series around $z = 0$.
 		
 		\item Laurent series in the annulus $1 < |z| < 3$.
 		
 		\item Laurent series in the region $3 < |z| < \infty$.
 	\end{enumerate}
 	
 	\item Expand in Laurent series in region indicated: $e^{\frac{1}{z - 1}}, |z| > 1$.
 	
 	\item Let $f(z):= \frac{1}{(z - 1)(z - 2)}$. Write $f(z)$ as a Laurent series centered at $z = 0$ which converges on the annulus $1 < |z| < 2$.
 	
 	\item Let $f(z) = \frac{1}{z^2 (e^z - e^{-z})}$, $0 < |z| < \pi$. Compute the first three non-zero terms of the Laurent expansion of $f(z)$ in $0 < |z| < \pi$.
 	
 	\item Prove that if $f$ is analytic on the region $U$ (open and simply connected), $z_0 \in U$, and $f'(z_0) = 0$, then $f$ is not one-to-one in any neighborhood of $z_0$.
 	
 	\item Let $\{a_n\}$ be a sequence of complex numbers. Assume that $f(z) = \sum_{n = 0}^{\infty} a_n z^n$ converges for all $z$ satisfying $|z| \leq r$. Prove that if $|a_1| > \sum_{n = 2}^{\infty} n |a_n| r^{n - 1}$, then $f$ is an injective function on the disc $|z| \leq r$.
 	
 	\item Prove that if $w = f(z)$ is holomorphic in the disc $D(0, 2)$, then $w = f(z)$ has a Taylor expansion centered at $z_0 = 0$ which converges for $|z| \leq 1$.
 	
 	\item Assume that $\{f_n(z)\}_{n = 1}^{\infty}$ is a sequence of analytic functions defined on the region $\Omega$, such that $\lim_{n \to \infty} f_n(z) = f(z)$ uniformly on compact subsets of $\Omega$. Show that $f(z)$ is analytic in $\Omega$ and that $\lim_{n \to \infty} f_n'(z) = f'(z)$ uniformly on compact subsets of $\Omega$.
 	
 	\item Classify the singularities at $z = 0$ of the following functions $f(z)$ and find their residue.
 	
 	\begin{enumerate}
 		\item $f(z) = \frac{1}{z}$
 		
 		\item $f(z) = z \cos \left( \frac{1}{z} \right)$
 		
 		\item $f(z) = z^{-3} \csc(z^2)$
 	\end{enumerate}
 	
 	\item Let $f(z):= \frac{e^z}{(z - 1)^4}$.
 	
 	\begin{enumerate}
 		\item Classify all of the singularities and find the associated residues.
 		
 		\item Determine the Laurent expansion of $f$ centered at $z = 1$.
 		
 		\item If $C$ denotes the positively oriented circle of radius 2 centered at $z = 0$, evaluate $\oint_C f(z) dz$.
 	\end{enumerate}
 	
 	\item Classify the singularities (including the point at $\infty$) and find the residues for
 	
 	\begin{enumerate}
 		\item $f(z) = \sin \left( \frac{1}{z} \right)$
 		
 		\item $f(z) = \frac{\sin(z^2)}{z^7}$
 		
 		\item $f(z) = \frac{1}{z^2} \cot z$
 	\end{enumerate}
 	
 	\item Classify all of the singularities and find the associated residues for each of the following functions:
 	
 	\begin{enumerate}
 		\item $\frac{(z + 3)^2}{z}$
 		
 		\item $\frac{e^{-z}}{(z - 1)(z + 2)^2}$
 	\end{enumerate}
 	
 	\item Classify all of the singularities and find the associated residues for each of the following functions:
 	
 	\begin{enumerate}
 		\item $z \cos(2z)$
 		
 		\item $\frac{z^3 - z^2 + 2}{z - 1}$
 	\end{enumerate}
 	
 	\item Classify the singularities of the following functions in the extended complex plane $\mathbb{C} \cup \{\infty\}$, and find the singular part at each of the isolated singular points:
 	
 	\begin{enumerate}
 		\item $\frac{1 - \cos z}{z^4}$
 		
 		\item $\sqrt{1 - \sin z}$
 		
 		\item $\frac{1 - z^3}{1 - z^2}$
 	\end{enumerate}
 	
 	\item Expand (if possible) in Laurent series in the indicated region:
 	
 	\begin{enumerate}
 		\item $e^{1/(x - 1)}$ for $|z| > 1$
 		
 		\item $\frac{1}{(z - a)(z - b)}$ for (i) $0 < |a| < |z| < |b|$ and (ii) $|a| < |b| < |z|$
 		
 		\item $\log \left( \frac{1}{1 - z} \right)$ for $|z| > 1$
 	\end{enumerate}
 	
 	\item Let $f(z) = \frac{1}{z - 1} + \frac{1}{(z - 2)^2}$. Expand $f(z)$ in:
 	
 	\begin{enumerate}
 		\item Taylor series in $|z| < 1$
 		
 		\item Laurent series in $1 < |z| < 2$
 	\end{enumerate}
 	
 	\item Find the Laurent expansion of $f(z) = (1 - z^2) e^{1/z}$ around $z = 0$. Determine its annulus of convergence and the residue of $f(z)$ at $z = 0$.
 	
 	\item 
 	
 	\begin{enumerate}
 		\item Find the first 3 non-vanishing terms of the Taylor series expansion of $\tan z$ around the origin.
 		
 		\item Also, find its radius of convergence.
 	\end{enumerate}
 	
 	\item Expand the function $f(z) = \frac{z^3 + 2z - 4}{z}$ in a power series around $z = 1$ and give its radius of convergence.
 	
 	\item Compute the radius of convergence of the following:
 	
 	\begin{enumerate}
 		\item $\sum_{n = 1}^{\infty} \frac{(3n)!}{(3n)^{3n}} z^n$
 		
 		\item $\sum_{n = 0}^{\infty} [3 + (-1)^n]^n z^n$
 		
 		\item The Taylor series around zero for the function $\frac{z}{e^z - 1}$
 	\end{enumerate}
 	
 	\item Compute the radius of convergence of the following:
 	
 	\begin{enumerate}
 		\item $\sum_{n = 1}^{\infty} \frac{(2n)!}{n^{2n}} z^n$
 		
 		\item $\sum_{n = 0}^{\infty} (n + a^n) z^n$ where $a \in \mathbb{C}$
 		
 		\item The Taylor series around zero for the function $z \cot z$
 	\end{enumerate}
 	
 	\item Determine the three Laurent series around 0 of the function $f(z) = \frac{1}{(z - 1)(z - 2)}$ in the three regions $|z| < 1$, $1 < |z| < 2$, and $|z| > 2$, respectively.
 	
\end{enumerate}

\newpage 
\section{Sequences of Analytic Functions}

\begin{enumerate}
	\item (CUNY, Fall 2005)
	Let $D$ be the closed unit disc. 
	Let $g_n$ be a sequence of analytic functions converging uniformly to $f$ on $D$. 
	\begin{enumerate}
		\item Show that $g_n'$ converges.
		\item Conclude that $f$ is analytic.
	\end{enumerate}

(Hint/Discussion: Normally, taking derivatives makes things numerically behave worse and integration makes things nicer. What is nice about complex analysis is that integration and differentiation are the same thing. Here is the hint now: use the integral formula for derivatives to get this done (I think). A basic philosophical point here is that differentiation of holomorphic functions is actually easy because it is secretely integration. )

	\item 
	Here is a first example of an analytic continuation ``from the wild''. 
	\begin{enumerate}
		\item Show that the Riemann Zeta function 
		$$ \zeta(z):=\sum_{n\geq 1} \frac{1}{n^z}$$
		converges for $\Re z > 1$ and is analytic on this domain. (You need to use the ``analytic convergence theorem'', which states that a uniform limit of analytic functions is analytic. This is just a slight generalization of the previous problem.)
		\item (Whittaker and Watson, 2.8, problem 10) 
		\begin{enumerate}
			\item Show that when $\Re s>1$,
			$$ \sum_{n=1}^{\infty} \frac{1}{n^s} = \frac{1}{s-1} + \sum_{n=1}^{\infty} \left[ \frac{1}{n^s} + \frac{1}{s-1} \left( \frac{1}{(n+1)^{s-1}} - \frac{1}{n^{s-1}} \right) \right]$$
			\item Show that the series on the right converges when $0<\Re s<1$. (This means the series above gives us access to the interesting part of the Riemann-Zeta function. Hint: $ \int_n^{n+1} x^{-s}dx = \frac{(n+1^{-s+1}}{1-s} - \frac{n^{-s+1}}{1-s} $)
		\end{enumerate}
	\end{enumerate}

\end{enumerate}

\section{Liouville's Theorem}

The proof of Liouville's Theorem is basically taking a limit in Cauchy's Integral Formula for the first derivative (if I don't understand what I mean here, this is your first exercise). 
There are variants of this proof which are featured in this problem.
The proof of the estimate of the partial sum for a power series expansion is based of expanding Cauchy's Integral Formula in a geometric series and then truncating the series. 
Many of the problems here were taken from University of New Mexico qualifying exams.

\begin{enumerate}
	\item Prove Liouville's Theorem: any bounded entire function is constant. 

	\item (New Mexico, not sure which year) %Problem 3
	Let $f$ be analytic on $\CC$. 
	Assume that $\max \lbrace \vert f(z) \vert : \vert z \vert = r \rbrace \leq M r^n$ for a fixed constant $M>0$, and a sequence of valued $r$ going to infinity. 
	Show that $f$ is a polynomial of degree less than or equal to $n$. 
	
	\item (New Mexico, not sure which year) 
	Let $f$ and $g$ be entire functions satisfying $\vert f(z) \vert \leq \vert g(z) \vert$ for $\vert z \vert \geq 100$. Assume that $g$ is not identically zero. Show that $f/g$ is rational. 
	
	\item Prove Goursat's theorem. Let $\gamma$ be a simple contour. If $f:\overline{\gamma^+} \to \CC$ is holomorphic (but whose derivative is not necessarily continuous) then 
	$$ \int_{\gamma} f(\zeta) d\zeta = 0. $$
	
	
	\item Let $f(z) = \sum_{n=0}^{\infty}a_nz^n$ and let $R$ be the radius of convergence (which is possibly infinite). Let $S_N(f)(z) = \sum_{n=0}^N a_n z^n$. 
	Show that for all $r<R$ and all $z \in \CC$ with $\vert z \vert < r$ we have 
	$$ \vert f(z) - S_N(f)(z) \vert \leq \frac{M(f,r)}{r - \vert z \vert} \frac{\vert z \vert^{N+1}}{r^{N}} $$ 
	where $ M(f,r) = \max_{\vert z \vert = r} \vert f(z) \vert. $
	
	\item (UIC, Spring 2016)
	Describe all entire functions such that $f(1/n) = f(-1/n) = 1/n^2$ for all $n\in \ZZ$.
	

		\item 
		
		\begin{enumerate}
		\item State and prove a form of the maximum principle.
		 \item State Schwarz's lemma and give a proof.
		\end{enumerate}
		

		\begin{enumerate}
			\item Deduce the fundamental theorem of algebra from Liouville's Theorem.
		\end{enumerate}

		
		\item Let $f$ be analytic in $\mathbb{C}$. Assume $\max\{|f(z)|:|z|=r\}\le Mr^{n}$ for a fixed constant $M>0$,
		and a sequence of values of $r$ going to infinity.
		Show that $f$ is a polynomial of degree less than or equal to $n$.
		
		\item Assume that $f(z)$ is an entire function with
		
		\[\lim_{|z|\rightarrow\infty}\frac{f(z)}{z^{2}}=0.\]
		
		Prove that $f(z)$ must be linear, that is $f(z)=a+bz,$ with $a, b\in \mathbb{C}$. Please provide all the details.
		
		\item Let $f(z)$ and $g(z)$ be entire functions. Show that if $f\circ g(z)$ is a polynomial then both $f(z)$ and $g(z)$ are polynomials.
		
		\item Let $f(z)$ and $g(z)$ be entire functions satisfying $|f(z)|\le 10|g(z)|$ for all $z\in\mathbb{C}$.
		
		Does it follow that there exists $\lambda\in\mathbb{C}$ with $f(z)=\lambda g(z)$ for all $z\in\mathbb{C}$?
		Give a proof or a counterexample.
		
		\item Assume that $w=f(z)$ is an entire function, and that $a$ and $b$ are two positive constants so that $f(z)$ satisfies $|f(z)|\le a+b|z|^{2}$ for all $z\in \mathbb{C}$. Prove that $f(z)$ is a polynomial of degree no larger than two.
		
		\item 
		
		\begin{enumerate}
			\item Let $f$ and $g$ be entire functions satisfying $|f(z)|\le|g(z)|$ for $|z|\ge100$. Assume $g$ is not identically zero.
		Show $f/g$ is rational.
		
		\item Let $u$ be harmonic in $\mathbb{C}$ and $u(x,y)\ge-2$ for all $x+iy\in \mathbb{C}$. Show $u$ is constant in $\mathbb{C}$.
		\end{enumerate}
		
		\item $\varphi(z)=|f_{1}(z)|^{2}+|f_{2}(z)|^{2}+\cdots+|f_{n}(z)|^{2}$.
		
		\begin{enumerate}
			\item Show that $\varphi(z)$ is harmonic on the domain only if all the functions $f_{k}(z)$ $(k=1,2,\ldots,n)$ reduce to constant functions.
			\item Show that $\varphi(z)$ has no local maximum unless all the functions $f_{k}(z)$ $(k=1,2,\ldots,n)$ reduce to constant functions.
		\end{enumerate}
		
		\item  If $u$ is harmonic and bounded in $0<|z|<\rho$, show that the origin is a removable singularity in the sense that $u$ becomes harmonic in $|z|<\rho$ when $u(0)$ is properly defined.
		

		
		\item
		\begin{enumerate} 
			\item Find a bound for the modulus of the integral shown below:
		
		\[\int_{\gamma}\sin^{2}(z)\,dz,\]
		
		where $\gamma$ is the simple contour $\gamma(t)=(1-t)+i\pi t$ and $0\le t\le1$.
		
		\item Evaluate exactly the modulus of the integral in (a).
		\end{enumerate}
		
		\item Let $f(z)$ be an entire function for which the real part $\Re f(x+iy)=u(x,y)$ is a bounded function.
		Does it follow that $f(z)$ is a constant function? Give a proof or a counterexample.
		
		\item Give two distinct harmonic functions on $\mathbb{C}$ that vanish on the entire real axis. Why is this not possible for analytic functions?
		
		\item Let $P_{1},P_{2},\ldots,P_{n}$ be arbitrary points of a plane and let $\overline{PP_{k}}$ denote the distance between $P_{k}$ and a variable point $P$. If $P$ is confined to the closure of a bounded domain $D$, show that the product $\prod_{k=1}^{n}\overline{PP_{k}}$ attains its maximum on the boundary of $D$.
		
		\item Let $f:\mathbb{C}\to\mathbb{C}$ be entire, and set $g(z):=f(1/z)$.
		Prove that $f$ is a polynomial if and only if $g(z)$ has a pole at $z=0$.
	\end{enumerate}
	

\section{Riemann Extension Theorem}
For functions which are analytic in some punctured neighborhood and which are bounded there is a natural way to extend the function to the point. This again uses the Cauchy integral formula and is another nice part of complex analysis.

\begin{enumerate}
	\item 
\begin{enumerate}
	\item Prove the Riemann Extension Theorem: Let $U\subset \CC$ be a region containing a point $z_0$. 
	Let $f \in \hol(U\setminus \lbrace z_0 \rbrace)$.
	If $f$ is bounded on $U$ show that there exists a unique $\widetilde{f} \in \hol(U)$ such that $\widetilde{f}\vert_{U \setminus \lbrace z_0 \rbrace} = f \in \hol(U \setminus \lbrace z_0 \rbrace )$.
	\item Recall that a morphism of topological spaces $f: X \to Y$ is ``proper'' if and only if the inverse image of every compact set is compact. 
	Show that an analytic map $f:\CC \to \CC$ is proper if and only if for all $z_j \to \infty$ we have $f(z_j) \to \infty$. 
	\item Show that the only proper maps $f:\CC \to \CC$ are polynomials. (see page 27 of McMullen, you need to consider the function $g(z) = 1/f(1/z)$ and show that $g(z) = z^n g_0(z)$ where $g_0(z)$ is analytic and non-zero. This will allows you to conclude $\vert g(z) \vert > c\vert z \vert^n$ for some $n$ which will allows you to conclude behavious about the growth of $f(z)$ as $z \to \infty$. )
\end{enumerate}
\end{enumerate}


\newpage
\section{Topological Things}
I collected a bunch of topological exercises here. 

	Background: 
\begin{itemize}
	\item Let $X$ and $Y$ be topological spaces. We define the topology on $X\times Y$ to be the smallest topology such that the projection maps $\pi_X: X \times Y \to X$ and $\pi_Y: X\times Y \to Y$ are continuous (this means the open sets are generated by sets of the form $U \times Y$ or $X\times V$ for $U\subset X$ open or $X \times V$ for $V\subset Y$ open. 
	\item A topological space $X$ is \textbf{compact} if every open cover has a finite subcover. An open cover is just a union of open sets that equal $X$.
	\item A \textbf{proper map} is a morphism of topological spaces such that the inverse image of compact sets is compact. 
\end{itemize}

Side Remark: The third condition is interesting because Grothendieck realized we can use it to extend this definition to categories other than topological spaces. In particular to the category of ``schemes''.

\begin{enumerate}
	\item Let $U \subset \CC$ be a connected open set. Consider $U\subset \CC$ with the subspace topology (open subset of $U$ are the intersection of open subsets of $\CC$ with $U$ and closed subset are closed subset of $\CC$ intersected with $U$). Show that the only subset of $U$ which are open, closed and nonempty is $U$ itself. 
	
	\item (Green and Krantz, Ch 11) A subset $S \subset \RR^n$ is \textbf{path connected} if for all $a,b \in S$ there exists a continuous $\gamma: [0,1] \to S$ such that $\gamma(0) = a$ and $\gamma(1) = b$. 
	
	Let $U$ be an open subset of $\RR^n$. Show that $U$ is path connected if and only if $U$ is connected. (Hint: show that the collection of path connected elements is open and closed. Also, you can use that the only nonempty open and closed subset of a connected open set is the entire set itself. )
	\item 
	Show that the following conditions are equivalent for a topological space $X$:
	\begin{enumerate}
		\item For all $a,b \in X$ there exists open sets $U \owns a$ and $V \owns b$ with $U \cap V = \emptyset$. 
		\item For all $a,b\in X$, if every neighborhood of $a$ intersects every neighborhood of $b$ then $a = b$.
		\item The diagonal map $X \to X\times X$ given by $x\mapsto (x,x)$ is proper. 
		\item The diagonal subset is closed. 
	\end{enumerate}
	
	
	If any of these conditions hold we call the topological space \textbf{separated} or \textbf{hausdorff}. (Hint: You should use the fact that a morphism $f$ is proper if and only if $f$ is closed and the inverse image of every point is compact.)
	
\end{enumerate}


\section{Harmonic Functions}

Let $u(x+iy)=u(x,y)$ be a real valued harmonic function on some region $U \subset \CC$.
A \textbf{harmonic conjugate} is a function $v(x,y)$ such that $f(x+iy):= u(x,y) + i v(x,y)$ is holomorphic.

\begin{enumerate}
			\item Assume that $w=f(z)=u(z)+iv(z)$ is an analytic function mapping a domain $D$ in the $z$-plane onto a domain $D^{\prime}$ in the $w$-plane.
	If $\phi(u,v)$ is a harmonic function in $D^{\prime}$, show that the function $\Phi(x,y)=\phi(u(x,y),v(x,y))$ is harmonic in $D$.
	
	\item Answer only one of the following questions:
	
	\begin{enumerate} 
		\item Prove that if $u(z)$ is a non-constant harmonic function in the domain $D$, then $u(z)$ does not have a local maximum in $D$.
		
		\item Let $u(x,y)$ be a harmonic function on the entire plane, and let $v(x,y)$ be a harmonic conjugate of $u(x,y)$.
		Assume that $u(x,y)\le v^{2}(x,y)$ for all $(x,y)\in \mathbb{C}$. Prove that both $u(x,y)$ and $v(x,y)$ must be constant.
	\end{enumerate}
	
	\item 
	\begin{enumerate}
		\item  Show that $u(x,y)=(x+1)y$ is harmonic in the entire plane.
		\item  Find a harmonic conjugate $v(x,y)$ of $u(x,y)$.
		\item Give explicitly an analytic function $w=f(z)$ with $u=\Re f$ and $v=\Im f$.
	\end{enumerate}
	
	
	\item 
	\begin{enumerate}
		\item Show that $u(x,y)=x^{3}-3xy^{2}+y^{2}-x^{2}$ is harmonic in the entire plane.
		\item Find a harmonic conjugate $v(x,y)$.
		\item Give explicitly an analytic function $w=f(z)$ with $u=\Re f$ and $v=\Im f$.
	\end{enumerate}
	
	\item  A complex-valued function $f=U+iV$ is said to be harmonic on a domain $D\subset\mathbb{C}$ if $U$ and $V$ are harmonic on $D$. Show that $f$ is holomorphic on $D$ if and only if both $f$ and $zf$ are harmonic on $D$.
	
	\item  Let $u(x,y)$ be the bounded harmonic function in the upper half-plane $\{z=x+iy\in \mathbb{C}|y>0\}$ that has the boundary value
	
	\[u(x,0)=\operatorname{sgn} x=\begin{cases} -1, & \text{if } x<0, \\ 1, & \text{if } x>0. \end{cases}\]
	
	Find a harmonic conjugate $v(x,y)$ of $u(x,y)$.
	
	\item  Show that $u(x,y) = u(z)$ has a harmonic conjugate locally. (Hint: Use the fundamental theorem of line integrals $v(\vec{P}) -v(\vec{Q})  = \int_{C} \nabla v \cdot d\vec{r}$ if $C$ is a path starting a $\vec{Q}$ and ending at $\vec{P}$)

	\item 
	Find all of the harmonic conjugates of $u(x,y) = x^3 - 3xy^2 + 2x$. 
	
	\item Let $f(z) = u(z) + i v(z)$ be analytic. 
	Show that the level sets of $u(z)$ and $v(z)$ are orthogonal.
	
	\item (New Mexico, Summer 2000)
	Show that the pullback of a harmonic function by an holomorphic map is harmonic (what these words means is explained below).
	Assume that $w = f(z) = u(z)+iv(z)$ is holomorphic map $f:D \to D' \subset \CC$. We consider $D$ in the $z$-plane to a domain $D'$ in the $w$-plane. 
	If $\phi$ is harmonic on $D'$, show that 
	$$ \Phi(x,y) := \phi(u(x,y),v(x,y))$$
	is harmonic in $D$. 
	
	(The function $\Phi$ is called the pullback of $\phi$ by $f$. Sometimes in the literature these you will see the notation $f^*\phi$ for $\Phi$.)
		
	\item (New Mexico, not sure which year) 
	Let $f(z)$ and $g(z)$ be entire functions. 
	Show that if $f(g(z))$ is a polynomial then both $f(z)$ and $g(z)$ are polynomials.
	(Hint: this relates to the problem on properness from the previous homework).
	
	\item Find all entire functions $f(z)$ which satisfy $\Re f(z) \leq 2/\vert z \vert$ when $\vert z \vert > 1$. (Hint: Consider $e^{-f(z)}$ or $e^{f(z)}$. You will need the maximum modulus principle and Liouville's theorem.)
	
	\item Let $u(z)$ be a real valued harmonic function on a domain $D \subset \CC$ 


		\item Show that for all $D_r(z_0) \subset D$ we have 
		$$ u(z_0) = \frac{1}{2\pi} \int_0^{2\pi} u(z_0+ r e^{i\theta}) d\theta. $$ (Hint: use a harmonic conjugate)
		\item If $z_0 \in D$ has the property that there exists some $r>0$ with $D_r(z_0) \subset D$ and 
		$$ u(z_0) \geq u(z) $$
		for all $z \in D_r(z_0)$ then $u(z)$ is constant. 
		(Hint: Consider a function such that $f(z) = u(z)+iv(z)$ then consider the maximum of $e^{f(z)}$.)
		
		
			
		\item Let $u_0(\theta)$ be a continuous $2\pi$-periodic function. 
			Let $D$ be a disc of radius $r$. 
			The Dirichlet boundary value problem asks to find a function $u(x,y)$ such that:
			$$ \begin{cases}
			\frac{\partial^2 u}{\partial x^2} + \frac{\partial^2 u}{\partial y^2} =0, & \mbox{ for $(x,y)\in D$ } \\
			u(e^{i\theta})= u_0(\theta), & 
			\end{cases}
			$$
			Show that convolution with the Poisson kernel 
			$$P_r(\theta) = \frac{1-r^2}{1-2r\cos(\theta) + r^2}$$
			gives a solution to this problem. 
			
	
	

	
\end{enumerate}

\section{Residue Integrals}
This is what I call the ``Residue Integral Gaunlet''. You should do all of these before your qualifying exam. 
Many of the integrals from from a course given by Vladimir Zakharov at University of Arizona. 
Some were taken from Whittaker and Watson. Some were taken from University of New Mexico qualifying exams.

\begin{enumerate}
		\item (Whittaker and Watson, 6.24,3)
	If $-1 < z < 3$ then 
	$$ \int_0^{\infty} \frac{x^z}{(1+x^2)^2}dx = \frac{\pi (1-z) }{4 \cos( \pi z/2) }$$
	
	\item (Whittaker and Watson, 6.21, Example 4)
	Let $a>b>0$ be real numbers. 
	Show that 
	$$ \int_0^{2\pi} \frac{d\theta}{(a+b\cos(\theta))^2} = \frac{2\pi a}{(a^2-b^2)^{3/2}}$$
	
	%$$ \int_0^{2\pi} \frac{d\theta}{(a+b\cos(\theta)^2)^2} = \frac{2\pi (2a+b)}{a^{3/2}(a^2-b^2)^{3/2}}$$
	
	\item (Whittaker and Watson, 6.23, 2) If $a>0$ and $b>0$ show that 
	$$ \int_{-\infty}^{\infty} \frac{x^4dx}{(a+bx^2)^4} = \frac{\pi}{16 a^{3/2}b^{5/2}}$$
	
	\item (Whittaker and Watson, 6.22, 1) Show that if $a>0$ then 
	$$ \int_0^{\infty} \frac{\cos(x)}{x^2+a^2}dx = \frac{\pi}{2a}e^{-a}. $$
	
	
	\item (Whittaker and Watson, 6.22)
	If the $\Re z >0$ then 
	$$ \int_0^{\infty} (e^{-t} - e^{-tz}) \frac{dt}{t} = \log z $$
	
	\item (Whittaker and Watson, 6.24,2) If $0 \leq z \leq 1$ and $-\pi< a \leq \pi$ then
	$$\int_0^{\infty} \frac{t^{z-1}}{t + e^{ia} }dt  = \frac{\pi e^{i(z-1)a}}{\sin(\pi z)} $$
	
	
	\item (Whittaker and Watson 6.24, 1, pg118)
	If $0<a<1$ show that 
	$$ \int_0^{\infty} \frac{x^{a-1}}{1+x}dx = \pi \csc a \pi $$
	
	
	\item (Whittaker and Watson, 6.24, 4)Show that if $-1<p<1$ and $-\pi<\lambda<\pi$ we have 
	$$ \int_0^{\infty} \frac{x^{-p} dx}{1 + 2x \cos(\lambda) + x^2 } = \frac{\pi}{\sin(p\pi)} \frac{\sin(p\lambda)}{\sin(\lambda)}$$
	
	\item  (Whittaker and Watson, 6.21, Example 3)
	Let $n$ be a positive integer. 
	Show that 
	$$ \int_0^{2\pi} e^{\cos(\theta)} \cos(n\theta - \sin\theta)d\theta = \frac{2\pi}{n!}$$
	
\item 
	
	Answer only one of the following questions:
	\begin{enumerate}
	\item Evaluate the integral $\int_{0}^{2\pi}\frac{d\theta}{1+\sin^{2}\theta}$.
	
	\item Evaluate the integral $\int_{0}^{\infty}\frac{x \sin x}{1+x^{2}}dx$.
	\end{enumerate}
	
\item
	Consider $I = \int_{\gamma}\frac{z^{2}dz}{1+z^{4}}$, where $\gamma$ is the contour shown below:
	
	[Insert image of contour here]
	
	\item Evaluate $I$ when $R < 1$.
	\item  Evaluate $I$ when $R > 1$.
	\item  Discuss the results obtained when $R \rightarrow \infty$.
	
\item 
	Choose a branch of $\sqrt{z^{2}-1}$ that is analytic on $\mathbb{C} \backslash \{z + 0i \mid -1 \leq z \leq 1\}$ and has the value $\sqrt{3}$ at $z = 2$. Evaluate $\int_{\gamma}\sqrt{z^{2}-1}dz$, where $\gamma$ is a contour (not specified).
	
\item Evaluate the following integrals
\begin{enumerate}
	\item \item  $\int_{0}^{\infty}\frac{\ln x}{4+x^{2}}dx$.
	\item  $\oint_{\gamma}\frac{e^{z}}{(z+1)(z-2i+1)}dz$, where $\gamma$ is the ellipse given by $\frac{x^{2}}{4} + y^{2} = 1$, with positive orientation (counterclockwise).
	Evaluate the following integrals:
	\item $\int_{0}^{\infty}\frac{\ln x}{4+x^{2}}dx$.
\end{enumerate}
		
\item 
	Define the function $F(z) = \int_{|\zeta| = z}\frac{d\zeta}{\zeta(\zeta-z)(\zeta-z+1)}$. Determine the limit of $F(z)$ as $z \rightarrow 2$ from:
	
\begin{enumerate}
	\item Inside the circle $|z| = 2$.
	\item Outside the circle $|z| = 2$.
	\item Is $F(z)$ continuous at $z = 2$?
\end{enumerate}
	

\item 
	
	Evaluate:
	
	\begin{enumerate}
	\item $\int_{|z|=2}\frac{z+6}{z^{2}-2z-3}dz$.
	
	\item $\int_{0}^{\pi}\frac{d\theta}{6-3\cos\theta}$.
	\end{enumerate}
	
\item 
	
	Let $\gamma \subset \mathbb{C}$ be the square centered at $z = 0$ with vertices at $z = \pm 2 \pm 2i$. Compute $\oint_{\gamma}\frac{z}{z^{3}+1}dz$, where $\gamma$ is traversed once in the counterclockwise direction.
	
\item 
	
	Evaluate $\int_{|z|=1}\frac{1-\cos z}{(e^{z}-1)\sin z}dz$.
	
\item 
	
	Given $f(z) = \frac{1}{z} - \frac{2}{z^{2}}$, find $\int_{C}z^{2}\exp(\frac{1}{z})f(z)dz$, where $C$ is the unit circle traversed counterclockwise.
	
\item 
	
	Evaluate the integral $\int_{0}^{\pi}\frac{d\theta}{a+\sin^{2}(\theta)}$, where $a > 0$.
	
\item 
	
	Evaluate the integral $\int_{0}^{\pi}\frac{dt}{5+4\cos t}$.
	
\item 
	
	Assume that $a$, $b$, $c$ are real numbers satisfying $ac - b^{2} > 0$. Prove using residues that $\int_{-\infty}^{\infty}\frac{dx}{ax^{2}+2bx+c} = \frac{\pi}{\sqrt{ac-b^{2}}}$.
	
\item 
	
	Show that $\int_{-\infty}^{+\infty}\operatorname{sech}^{2}(x)\cos(2x)dx = \frac{2\pi}{\sinh(\pi)}$.
	
\item 
	
	Compute the integral $I = \frac{1}{2\pi}\int_{-\infty}^{\infty}\frac{\sin x}{x(x^{2}+1)}dx$. Carefully justify all your steps.
	
\item 
	
	Evaluate the real integral $\int_{0}^{\infty}\frac{x^{2}}{(x^{2}+1)^{2}}dx$ and justify all steps.
	
\item 
	
	Use the theory of residues to evaluate the integral $\int_{0}^{\infty}\frac{\sqrt{x}dx}{x^{2}+4}$.
	
\item 
	
	Use the method of contour integration and the calculus of residues to evaluate the integral $\int_{0}^{\infty}\frac{x^{p}}{1+x^{2}}dx$, where $-1 < p < 1$. Draw the relevant contour and justify all steps.
	
\item 
	
	Compute $F(k) = \frac{1}{\sqrt{2\pi}}\int_{-\infty}^{\infty}\frac{e^{ikx}}{\pi^{2}+1}dx$, where $k \in \mathbb{R}$.
	
\item 
	
	Find the Fourier transform of the function $f(x) = e^{-x/4}$; i.e., find the function $\hat{f}(t)$ defined for all $t \in \mathbb{R}$ by $\hat{f}(t) = \frac{1}{\sqrt{2\pi}}\int_{-\infty}^{\infty}e^{-\frac{z^{2}}{2}} \cdot e^{-itz}dz$.
	
\item 
	
	Evaluate $\int_{-\infty}^{\infty}\frac{\exp(ikx)}{(1+x^{2})}dx$ by contour integration (assume $k > 0$ real).
	
\item 
	(a) $\int_{0}^{\infty}e^{-x^{2}}\cos(\lambda x)dx$.
	
	(b) $\int_{0}^{2\pi}\frac{d\theta}{1+\cos^{2}\theta}$.
	
	It may be helpful to know that $\int_{-\infty}^{+\infty}e^{-x^{2}}dx = \sqrt{\pi}$.
	
\item 
	
	Show by the method of complex contour integration that the following identities hold:
	\begin{enumerate}
	\item $\int_{0}^{\infty}\frac{\sin ax}{x(x^{2}+1)}dx = \frac{1-e^{-a}}{2}$.
	
	\item $\int_{0}^{\infty}\frac{dx}{\sqrt{x}(x+1)} = \pi$.
	\end{enumerate}
	
\item 
	
	Answer only one of the following questions:
	
	\begin{enumerate}
		\item Evaluate the integral $\int_{0}^{\infty}\frac{x^{\alpha-1}}{1+x}dx$, where $0 < \alpha < 1$.
		\item Evaluate the integral $\int_{0}^{\infty}\frac{x^{\alpha-1}}{1+x}dx$, where $0 < \alpha < 1$.
	\end{enumerate}


	
	\item Evaluate the integral $\int_{0}^{\infty}\frac{x \sin x}{(x^{2}+1)(x^{2}+4)}dx$.
	
	\item Evaluate the improper integral $\int_{-\infty}^{\infty}\frac{\cos(x)dx}{1+x^{4}}$. Include justifications for all steps in your calculation.
	
	\item Use the theory of residues to evaluate the integral $\int_{0}^{\infty}\frac{\sqrt{x}dx}{x^{2}+4}$.

	\item Evaluate $\int_{0}^{\infty}\frac{\sqrt{x}}{(1+x)^{3}}dx$ by contour integration.
	
	\item Evaluate the integral $\int_{-\infty}^{\infty}\frac{\sin^{3}x}{\pi^{3}}dx$.
	

	
	\item Show that $\int_{-\infty}^{+\infty}\frac{\cos x - \cos a}{x^{2}-a^{2}}dx = -\pi\frac{\sin a}{a}$, where $a \in \mathbb{R}^{+}$.
	
\end{enumerate}

\section{Rouche's Theorem and Argument Principal}

Many of the problems not listed were taken from University of New Mexico qualifying exams. 

\begin{enumerate}
	
	\item (New Mexico, Jan 1997) How many roots does $p(z) = z^4 + z + 1$ have in the first quadrant?
	
	\item (New Mexico, Aug 1993)
	How many roots does $e^z - 4z^n +1 =0$ have inside the unit disc $\vert z \vert < 1$?
	
	  \item For each $n\in\mathbb{N}$ set
	$f_{n}(z):=\sum_{j=1}^{n}\frac{z^{-j}}{j!}$. For a given $\rho>0$, show that there is an $N(\rho)$
	such that if $n>N(\rho)$, then all of the zeros of $f_{n}(z)$ lie within $D(0,\rho)$.
	
	\item Find all entire functions that satisfies the Lipschitz condition on $\mathbb{C}$. A function $f(z)$ is said to satisfy the Lipschitz condition on $\mathbb{C}$ if there exists a positive constant $M$ such that $|f(z_{1})-f(z_{2})|\le M\cdot|z_{1}-z_{2}|$ for all $z_{1}$, $z_{2}\in \mathbb{C}$.
	
	\item Let $p_{n}(z)=1+\frac{z}{1!}+\frac{x^{2}}{2!}+\cdot\cdot\cdot+\frac{z^{n}}{n!}$ Prove that for every $R>0$ there exists a positive integer $n(\hat{R})$ such that all roots of $p_{n}(z)=0$ for $n\ge n(R)$ belong to the set $\{z\in\mathbb{C}||z|>R\}$.
	
	\item For each $n\in\mathbb{N}$ set $p_{n}(z):=\sum_{j=0}^{n}(-1)^{j}\frac{z^{-2j}}{j!}$.
	\begin{enumerate}
		\item For each fixed $n\in\mathbb{N}$, show that $p_{n}(z)=0$ has precisely $2n$ solutions.
		\item For a given $\rho>0$, show that there is an $N(\rho)$ such that if $n>N(\rho)$, then all of the zeros of $p_{n}(z)$ lie within $D(0,\rho)$.
	\end{enumerate}
	
	\item Answer only one of the following questions:
	\begin{enumerate}
		\item State and prove Rouche's theorem.
		\item State and prove the argument principle.
	\end{enumerate}
	
	\item Assume $f(z)$ is analytic inside and on a contour $C$ and $b$ is a complex number such that $f(z)=b$ for some on $C$. What is the significance of this?
	
	\item State Rouche's theorem and use it to determine how many roots of the polynomial $z^{4}+5z+3$ lie inside:
	\begin{enumerate}
		\item the unit disc.
		\item the annulus $1<|z|<2$.
	\end{enumerate}
	
	\item State Rouche's theorem and find the number of zeros of the function $f(z)=2z^{5}+7z^{3}+z^{2}-3$ in the annulus $1\le|z|<2$.
	
	\item Show that the equation $(z-2)^{2}=e^{-z}$ has two distinct roots in the disc $|z-2|\le1$.
	
	\item State and prove Rouche's theorem. Use Rouche's theorem to show that there is $\epsilon_{0}>0$ so that for $0<\epsilon<\epsilon_{0}$ the equation $z^{3}-\epsilon z-1=0$ has three distinct roots.
	
	\item State some version of Rouche's theorem, and then use it to show that $f(z):=ze^{3-z}-1$ has only one real zero in $D(0,1)$.
	
	\item State some version of Rouche's theorem, and then use it to show that all of the zeros for $f(z):=z^{8}-4z^{3}+10$ lie in the annulus $D(0,2)\backslash\overline{D}(0,1)$.
	
	\item Prove the argument principle. How many roots does the polynomial $p(z)=z^{4}+z+1$ have in the first quadrant?
	
	\item State Rouche's theorem and use it to show that all zeros of the polynomial $p(z)=z^{4}+6z+3$ lie in the circle $|z|<2$. How many zeros of $p(z)$ lie in the annulus $1<|z|<2$?
	
	\item Let $f(z)=z^{10}+\frac{1}{2}z^{6}+\frac{1}{100}exp(z^{5})$.
	\begin{enumerate}
		\item State Rouche's theorem.
		\item Show that $f(z)$ has no zeros on $|z|=1$.
		\item How many zeros does $f(z)$ have inside $\{z:|z|=1\}$? Justify your answer.
	\end{enumerate}
	
	\item Show that $z^{5}-15z+1=0$ has one root in the disc $|z|<\frac{1}{8}$ and four roots in the annulus $\frac{3}{2}<|z|<2$.
	
	\item Show that $z^{5}+15z+1=0$ has one root in the disc $|z|<\frac{3}{2}$, four roots in the annulus $\frac{3}{2}<|z|<2$.
	
	\item Let $f(z)=z^{4}-5z+1$.
	\begin{enumerate}
		\item How many zeros does $f(z)$ have in the disc $\{z\in \mathbb{C}:|z|<1\}$?
		\item How many zeros does $f(z)$ have in the annulus $1<|z|<2$?
	\end{enumerate}
	
	\item Find the number of zeroes of the equation $e^{z}-4z^{n}+1=0$ in the unit disc $D=\{z\in \mathbb{C}:|z|<1\}$.
	
\end{enumerate}


\section{Conformal Maps}
\begin{enumerate}
\item (New Mexico, Summer 1999)
Let $\mathcal{H}$ be the upper half complex plane $\mathcal{H} = \lbrace z \in \CC : \Im z > 0 \rbrace$.
Let $D$ be the unit disc $D = \lbrace w : \vert w \vert < 1 \rbrace$. 
Show that the map $f(z) = w = (z-i)/(z+i)$ defines a bijection $\mathcal{H} \to D$.
	
	
\item Find the points where $w = f(z)$ is conformal if 
\begin{enumerate}
	\item $w = \cos(z)$
	\item $w = z^5 - 5z$
	\item $w = 1/(z^2+1)$
	\item $w = \sqrt{z^2+1}$.
\end{enumerate}

\item Find a conformal map of the strip $0 < \Re z < 1$ onto the unit disc $\vert w \vert < 1$ in such a way that $z=1/2 $ goes to $ w=0$ and $z = \infty$ goes to $w=1$.

\item Find the M\"obius transformation that maps the left have plane $\lbrace z \in \CC: \Re z < 1 \rbrace$ to the unit disc $\lbrace w \in \CC : \vert w \vert < 1$ and has $z=0$ and $z=1$ as fixed points.

\item Find a conformal map from the following regions onto the unit disc $D = \lbrace z : \vert z \vert < 1 \rbrace$
\begin{enumerate}
	\item $A = \lbrace z: \vert z \vert < 2,  \Arg(z) \in (0,\pi/4) \rbrace $
	\item $B = \lbrace z: \Re(z) >2 $
	\item $C = \lbrace z: -1<\Re(z)<1 \rbrace$
	\item $D' = \lbrace z: \vert z \vert < 1 \mbox{ and } \Re z < 0\rbrace $ 
\end{enumerate}

\item Let $D$ be the unit disc. Let $f: D \to D$ be a conformal map. 
\begin{enumerate}
	\item If $f(0) = 0$ show that $f(z) = \omega z$ for some $\omega \in \partial D$. 
	\item If $f(0) \neq 0$ show that there exists some $a \in D$ and $\omega \in \partial D$ such that 
	$$ f(z) = \omega \frac{z - a}{1 - \overline{a} z}.$$
\end{enumerate}

\item 
\begin{enumerate}
	
	\item Show that $\PSL_2(\ZZ)$ is generated by $S(z) = -1/z$ and $T(z) = z+1$ and hence has the presentation
	$$ \langle S, T : S^2 = 1, (ST)^3 = 1 \rangle. $$
	
	\item Show that a fundamental domain\footnote{A fundamental domain for an action $\Gamma \times X \to X$ is a closed subset $\Omega \subset X$ such that 
		\begin{enumerate}
			\item $X = \bigcup_{\gamma \in \Gamma} \gamma(\Omega)$
			\item For all $\gamma \neq 1$ the set $\gamma(\Omega) \cap \Omega$ has empty interior.
		\end{enumerate}
		Note that this definition is different from what I had originally said in class. We had our fundamental domains have the property that $\gamma(\Omega) \cap \Omega = \emptyset.$ Unfortunately, as this example shows, we can't always arrange for this.   
	} for this action is the complement of the unit disc in a vertical strip of length 1 centered around zero in the upper half plane. In other words
	$$ \Omega =  \lbrace z: \vert z \vert \geq 1 \mbox{ and } -1/2\leq \Re(z) \leq 1/2 \rbrace$$ is a fundamental domain for this action.  
	
	\item 
	Show that the following points are fixed points of $\overline{\Omega}$ with the following stabilizers:
	\begin{enumerate}
		\item $\Stab(i) = \lbrace 1, S\rbrace$
		\item $\Stab(e^{2\pi i/2}) = \lbrace 1, ST, (ST)^2 \rbrace $
		\item $\Stab(e^{\pi i/3}) = \lbrace 1, TS, (TS)^2 \rbrace $
	\end{enumerate}
	(Note: this exercise gives you an example of an action that is not free.)
\end{enumerate}

\end{enumerate}


\section{Elliptic and Modular Functions} 

\begin{enumerate}
\item Show that 
$$ \wp_{\Lambda}(z) = \frac{1}{z^2} + \sum_{\lambda \in \Lambda^*} \left[ \frac{1}{(z-\lambda)^2} - \frac{1}{\lambda^2} \right] $$
is elliptic with period lattice $\Lambda$.

\item For a lattice $\Lambda \subset \CC$ and $m\geq 3$ define $G_m = G_m(\Lambda) =  \sum_{\lambda \in \Lambda\setminus \lbrace 0 \rbrace } \lambda^{-m}. $   
\begin{enumerate}
	\item Show that $ \wp(z) - \frac{1}{z^2} = \sum_{k=1}^{\infty} (k+1)G_{k+2} z^k.$\footnote{You may need to use that you can interchange some series. If $f_n(z) = \sum a_j^{(n)} z^j$ and $A_j = \sum_{n=0}^{\infty} a_j^{(n)} $ converges then $\sum_{n=0}^{\infty} f_n(z) = \sum_{j=0}^{\infty} A_j z^j$. } 
	\item Conclude that 
	$$ \wp'(z)^2 - 4 \wp(z)^3 + g_2 \wp(z) + g_3 = O(z^2),$$
	as $z \to 0$, which shows that $\wp'(z)^2 - 4 \wp(z)^3 + g_2 \wp(z) + g_3$ is analytic at the origin of $\CC$. 
	Here $g_2 = 60 G_4$ and $g_3 = 140 G_6$.
	\item Conclude that $\wp'(z)^2 - 4 \wp(z)^3 + g_2 \wp(z) + g_3$ is constant. (Hint: use that elliptic functions without poles are constant.)
	\item Show the constant in the previous number is zero.
\end{enumerate}

\item The zeros of $\wp(z)-c$ are simple with precisely double zeros at the points congruent to $\omega_1/2, (\omega_1+\omega_2)/2, \omega_2/2$. (Hint: what are the zeros of $\wp'(z)$ and what does this mean?)

\end{enumerate}



\section{Riemann Surfaces}
This uses some basic properties of Riemann Surfaces.

\begin{enumerate}
\item 
\begin{enumerate}
	\item Show that every automorphism of $\CC$ extends to an automorphism of $\PP^1$.
	\item Show that $\Aut(\CC):= \lbrace az+b : a \in \CC^{\times} \mbox{ and } b\in \CC \rbrace$ (This sometimes called the one dimensional affine linear group and is denoted $\mathrm{AL}_1(\CC)$.).
\end{enumerate}

\item Show that $\CC$ is not conformally equivalent to $D = \lbrace z \in \CC: \vert z \vert < 1 \rbrace$. 

\item Show that $\Aut(H) = \lbrace \frac{az+b}{cz+d}: a,b,c,d\in \RR \mbox{ and } ad-bc =1 \rbrace$ (This is sometimes called the two dimensional projective special linear groups with coefficients in $\RR$, and is denoted $\PSL_2(\RR)$).

\end{enumerate}

\section{Infinite Products}

\begin{enumerate}
	\item Show the Gauss formula for the Gamma function:
	$$ \Gamma(z) = \lim_{n\to\infty} \frac{n^z n!}{z(z+1)(z+2) \cdots (z+n)}. $$
	(Take the definition of the Gamma function to be from its product formula).
	\item Verify that $F(z) = \int_0^{\infty} t^{z-1}e^{-t}dt$ and $\Gamma(z)$ (via $1/\Gamma(z)$ being defined by the product formula) satify the hypotheses of Weilandt's Theorem. In particular that $F(z)$ and $\Gamma(z)$ are bounded when $1 <\Re z <2$. 
	
	\item Show that $\int_0^{2\pi} \log \vert 1 - e^{i\theta}\vert d\theta=0$.
	
	\item (New Mexico, Jan 2006)
	Consider $f(z) = \prod_{n=1}^{\infty}(1-z/n^3)$.
	What is the order of $f(z)$?
	
	\item Let $f(z) = \sum_{n=0}^{\infty} a_n z^n$ 
	be an entire function of finite order $\rho$. Show that 
	$$ \rho = \liminf_{n\to\infty} \frac{\log(n)}{\log \vert a_n \vert^{-1/n}}.$$
\end{enumerate}

\section{The Big Picard Theorem}

\begin{enumerate}	
	\item 
	\begin{enumerate}
		\item Prove the Casorati-Weiestrass Theorem: Let $f(z)$ is analytic in a punctured disc of radius $R$ at the origin. If $f(z)$ has an essential singularity at $z=0$ show that for every $r$ with $0<r<R$ the set $f(D_r(0)\setminus \lbrace 0 \rbrace)$ is dense in $\CC$. (This is a corollary of Big Picard).
		\item Let $p$ be a polynomial. Show that there exists infinitely many $z_j$ such that $p(z_j) = e^{z_j}$.
	\end{enumerate}
	
	\item 
	The following exercise is intended to introduce you to the $j$ function which plays a role in the proof of the Big Picard Theorem from class. 
	
	Let $H$ be the upper-half plane. 
	A \textbf{modular form} of weight $k$ and level $N=1$ is a function $f:H \to \CC$ such that  
	\begin{equation}
	f( \frac{az+b}{cz+d} ) = (cz+d)^{-2k} f(z).
	\end{equation}
	for all $\left[\begin{matrix}
	a & b \\
	c & d 
	\end{matrix} \right ]\in \SL_2(\ZZ)$. 
	
	\begin{enumerate}
		\item Let $M_k$ denote the collection of modular forms of weight $k$ and level 1. 
		Show that $M = \bigoplus_{k\geq 0} M_k$ is a graded ring (i.e. that $M_{k_1}M_{k_2} \subset M_{k_1+k_2}$. 
		
		\item Show that $G_{2k}(\frac{az+b}{cz+d}) = (cz + d)^{2k} G_{2k}(z)$ has weight $2k$ (Hint: check this on the generators of $\SL_2(\ZZ)$.)
		
		Using the first part conclude that the we have the following modular forms of the indicated weights:
		\begin{enumerate}
			\item $g_2(\tau) = 60 G_4(\tau)$, $k=4$
			\item $g_3(\tau) = 140 G_6(\tau)$, $k=6$
			\item $\Delta(\tau) = g_2(\tau)^3 - 27 g_3(\tau)^2$, $k=12$
			\item $j(\tau) = 1728 g_2(\tau)^3/\Delta(\tau)$, $k=0$
		\end{enumerate}
	\end{enumerate}
	
	\item Explain in words the ideas that go into the proof of Montel's Theorem in Green and Krantz (page 193). 
	How is Arzela-Ascoli used?
	
	\item Let $X=\CC^{\times} = \CC \setminus \lbrace 0 \rbrace$. 
	What is the universal cover of $X$? What is group of deck transformations for this cover? 
	
	
	\item Use Van Kampen's theorem to rigorously compute 
	$\pi_1(\mathbf{P}^1\setminus \lbrace p_1,\ldots, p_r\rbrace, z_0)$ for arbitrary $r$. (Hint: apply Van Kampen to open sets $U,V$ where $U\cap V$ is simply connected). 
	
\end{enumerate}



\end{document}
