\documentclass[a4paper,10pt]{article}
\usepackage[utf8]{inputenc}
\usepackage{amssymb}
\usepackage{amsmath}
\usepackage{fullpage}
\newcommand{\ZZ}{\mathbf{Z}}
\newcommand{\RR}{\mathbf{R}}
\newcommand{\CC}{\mathbf{C}}
\newcommand{\HH}{\mathbf{H}}
\renewcommand{\Re}{\operatorname{Re}}
\renewcommand{\Im}{\operatorname{Im}}
\newcommand{\hol}{\operatorname{hol}}
\newcommand{\id}{\operatorname{id}}

%opening
\title{}
\author{Dupuy --- Complex Analysis --- Spring 2017 --- Homework 03}
\date{} 


\begin{document}

\maketitle

\begin{enumerate}
\subsection*{Master's level}

\item Find all entire functions $f(z)$ which satisfy $\Re f(z) \leq 2/\vert z \vert$ when $\vert z \vert > 1$. (Hint: Consider $e^{-f(z)}$ or $e^{f(z)}$. You will need the maximum modulus principle and Liouville's theorem.)

\item Let $u(z)$ be a real valued harmonic function on a domain $D \subset \CC$ 
\begin{enumerate}
\item A \textbf{harmonic conjugate} is a function $v(x,y)$ such that $f(x+iy):= u(x,y) + i v(x,y)$ is holomorphic. Show that $u(x,y) = u(z)$ has a harmonic conjugate locally. (Hint: Use the fundamental theorem of line integrals $v(\vec{P}) -v(\vec{Q})  = \int_{C} \nabla v \cdot d\vec{r}$ if $C$ is a path starting a $\vec{Q}$ and ending at $\vec{P}$)
\item Show that for all $D_r(z_0) \subset D$ we have 
 $$ u(z_0) = \frac{1}{2\pi} \int_0^{2\pi} u(z_0+ r e^{i\theta}) d\theta. $$ (Hint: use a harmonic conjugate)
\item If $z_0 \in D$ has the property that there exists some $r>0$ with $D_r(z_0) \subset D$ and 
 $$ u(z_0) \geq u(z) $$
for all $z \in D_r(z_0)$ then $u(z)$ is constant. 
(Hint: Consider a function such that $f(z) = u(z)+iv(z)$ then consider the maximum of $e^{f(z)}$.)

\end{enumerate}



\item Let $u(x+iy)=u(x,y)$ be a real valued harmonic function. 
A \emph{harmonic conjugate} is a function $v(x,y)$ such that $f(x+iy):= u(x,y) + i v(x,y)$ is holomorphic. 

Find all of the harmonic conjugates of $u(x,y) = x^3 - 3xy^2 + 2x$. 


\item (Green and Krantz, Ch 11) A subset $S \subset \RR^n$ is \textbf{path connected} if for all $a,b \in S$ there exists a continuous $\gamma: [0,1] \to S$ such that $\gamma(0) = a$ and $\gamma(1) = b$. 

Let $U$ be an open subset of $\RR^n$. Show that $U$ is path connected if and only if $U$ is connected. (Hint: show that the collection of path connected elements is open and closed. Also, you can use that the only nonempty open and closed subset of a connected open set is the entire set itself. )


\subsection*{Ph.D. level}


\item (New Mexico, not sure which year) 
Let $f(z)$ and $g(z)$ be entire functions. 
Show that if $f(g(z))$ is a polynomial then both $f(z)$ and $g(z)$ are polynomials.
(Hint: this relates to the problem on properness from the previous homework).

\item 
Show that the following conditions are equivalent for a topological space $X$:
  \begin{enumerate}
  \item For all $a,b \in X$ there exists open sets $U \owns a$ and $V \owns b$ with $U \cap V = \emptyset$. 
  \item For all $a,b\in X$, if every neighborhood of $a$ intersects every neighborhood of $b$ then $a = b$.
  \item The diagonal map $X \to X\times X$ given by $x\mapsto (x,x)$ is proper. 
  \item The diagonal subset is closed. 
  \end{enumerate}


If any of these conditions hold we call the topological space \textbf{separated} or \textbf{hausdorff}. (Hint: You should use the fact that a morphism $f$ is proper if and only if $f$ is closed and the inverse image of every point is compact.)

Background: 
	\begin{itemize}
	\item Let $X$ and $Y$ be topological spaces. We define the topology on $X\times Y$ to be the smallest topology such that the projection maps $\pi_X: X \times Y \to X$ and $\pi_Y: X\times Y \to Y$ are continuous (this means the open sets are generated by sets of the form $U \times Y$ or $X\times V$ for $U\subset X$ open or $X \times V$ for $V\subset Y$ open. 
	\item A topological space $X$ is \textbf{compact} if every open cover has a finite subcover. An open cover is just a union of open sets that equal $X$.
	\item A \textbf{proper map} is a morphism of topological spaces such that the inverse image of compact sets is compact. 
	\end{itemize}

Side Remark: The third condition is interesting because Grothendieck realized we can use it to extend this definition to categories other than topological spaces. In particular to the category of ``schemes''.

\end{enumerate}


\end{document}
