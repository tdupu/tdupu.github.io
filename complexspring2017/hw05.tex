\documentclass[a4paper,10pt]{article}
\usepackage[utf8]{inputenc}
\usepackage{amssymb}
\usepackage{amsmath}
\usepackage{fullpage}
\newcommand{\ZZ}{\mathbf{Z}}
\newcommand{\RR}{\mathbf{R}}
\newcommand{\CC}{\mathbf{C}}
\newcommand{\HH}{\mathbf{H}}
\renewcommand{\Re}{\operatorname{Re}}
\renewcommand{\Im}{\operatorname{Im}}
\newcommand{\hol}{\operatorname{hol}}
\newcommand{\Res}{\operatorname{Res}}
\newcommand{\ord}{\operatorname{ord}}
\newcommand{\SL}{\operatorname{SL}}
\newcommand{\PSL}{\operatorname{PSL}}
\newcommand{\Arg}{\operatorname{Arg}}
\newcommand{\Aut}{\operatorname{Aut}}
\newcommand{\Stab}{\operatorname{Stab}}

\newenvironment{sol}[1][Solution]{\begin{trivlist}
\item[\hskip \labelsep {\bfseries #1}]}{\end{trivlist}}
\newcommand{\qed}{\nobreak \ifvmode \relax \else
    \ifdim\lastskip<1.5em \hskip-\lastskip
    \hskip1.5em plus0em minus-.5em \fi \nobreak
    \vrule height0.75em width0.5em depth0.25em\fi}

%opening
\title{}
\author{Dupuy --- Complex Analysis --- Spring 2017 --- Homework 05}
\date{} 


\begin{document}

\maketitle



\begin{enumerate}

\subsection*{Conformal Maps}
\item Find the points where $w = f(z)$ is conformal if 
\begin{enumerate}
\item $w = \cos(z)$
\item $w = z^5 - 5z$
\item $w = 1/(z^2+1)$
\item $w = \sqrt{z^2+1}$.
\end{enumerate}

\item Find a conformal map of the strip $0 < \Re z < 1$ onto the unit disc $\vert w \vert < 1$ in such a way that $z=1/2 $ goes to $ w=0$ and $z = \infty$ goes to $w=1$.

\item Find the M\"obius transformation that maps the left have plane $\lbrace z \in \CC: \Re z < 1 \rbrace$ to the unit disc $\lbrace w \in \CC : \vert w \vert < 1$ and has $z=0$ and $z=1$ as fixed points.

\item Find a conformal map from the following regions onto the unit disc $D = \lbrace z : \vert z \vert < 1 \rbrace$
\begin{enumerate}
\item $A = \lbrace z: \vert z \vert < 2,  \Arg(z) \in (0,\pi/4) \rbrace $
\item $B = \lbrace z: \Re(z) >2 $
\item $C = \lbrace z: -1<\Re(z)<1 \rbrace$
\item $D' = \lbrace z: \vert z \vert < 1 \mbox{ and } \Re z < 0\rbrace $ 
\end{enumerate}

\item Let $D$ be the unit disc. Let $f: D \to D$ be a conformal map. 
\begin{enumerate}
\item If $f(0) = 0$ show that $f(z) = \omega z$ for some $\omega \in \partial D$. 
\item If $f(0) \neq 0$ show that there exists some $a \in D$ and $\omega \in \partial D$ such that 
 $$ f(z) = \omega \frac{z - a}{1 - \overline{a} z}.$$
\end{enumerate}

\item 
\begin{enumerate}

\item Show that $\PSL_2(\ZZ)$ is generated by $S(z) = -1/z$ and $T(z) = z+1$ and hence has the presentation
 $$ \langle S, T : S^2 = 1, (ST)^3 = 1 \rangle. $$

\item Show that a fundamental domain\footnote{A fundamental domain for an action $\Gamma \times X \to X$ is a closed subset $\Omega \subset X$ such that 
\begin{enumerate}
\item $X = \bigcup_{\gamma \in \Gamma} \gamma(\Omega)$
\item For all $\gamma \neq 1$ the set $\gamma(\Omega) \cap \Omega$ has empty interior.
\end{enumerate}
Note that this definition is different from what I had originally said in class. We had our fundamental domains have the property that $\gamma(\Omega) \cap \Omega = \emptyset.$ Unfortunately, as this example shows, we can't always arrange for this.   
 } for this action is the complement of the unit disc in a vertical strip of length 1 centered around zero in the upper half plane. In other words
  $$ \Omega =  \lbrace z: \vert z \vert \geq 1 \mbox{ and } -1/2\leq \Re(z) \leq 1/2 \rbrace$$ is a fundamental domain for this action.  

\item 
Show that the following points are fixed points of $\overline{\Omega}$ with the following stabilizers:
\begin{enumerate}
\item $\Stab(i) = \lbrace 1, S\rbrace$
\item $\Stab(e^{2\pi i/2}) = \lbrace 1, ST, (ST)^2 \rbrace $
\item $\Stab(e^{\pi i/3}) = \lbrace 1, TS, (TS)^2 \rbrace $
\end{enumerate}
(Note: this exercise gives you an example of an action that is not free.)
\end{enumerate}


\iffalse
\item (From John H. Conway's ``Sensual Quadratic Form'' Book) Let $p/q$ be a rational number. 
\begin{enumerate}
\item The \textbf{Ford Circle} at $p/q$ is a circle of radius $1/2q^2$ in the upper-half plane tangent to the real axis at $p/q$. 
Show that M\"obius transformations coming from $\PSL_2(\ZZ)$ take ford circles to ford circles.
\item Let $C_{p/q}$ be a Ford circle. Show that $C_{r,s}$ is tangent to $C_{p,q}$ if and only if they are neighbors $p/q$ and $r/s$ are neighbors in some Farey sequence.\footnote{A Farey Series is the sequence of rational numbers with denominators at most $d$. For example, the Farey Series of order 4 is $\ldots, 0/1. 1/4. 1/3. 1/2,  2/3. 3/4, 1/1, \ldots $}
\item Suppose $\alpha \in \RR$ is irrational. There are infinitely many rational $p/q$ such that 
 $$ \vert \alpha - p/q \vert < 1/2q^2. $$
\end{enumerate}
\fi

\subsection*{Elliptic Functions} 
\item Show that 
 $$ \wp_{\Lambda}(z) = \frac{1}{z^2} + \sum_{\lambda \in \Lambda^*} \left[ \frac{1}{(z-\lambda)^2} - \frac{1}{\lambda^2} \right] $$
is elliptic with period lattice $\Lambda$.

\item For a lattice $\Lambda \subset \CC$ and $m\geq 3$ define $G_m = G_m(\Lambda) =  \sum_{\lambda \in \Lambda\setminus \lbrace 0 \rbrace } \lambda^{-m}. $   
\begin{enumerate}
  \item Show that $ \wp(z) - \frac{1}{z^2} = \sum_{k=1}^{\infty} (k+1)G_{k+2} z^k.$\footnote{You may need to use that you can interchange some series. If $f_n(z) = \sum a_j^{(n)} z^j$ and $A_j = \sum_{n=0}^{\infty} a_j^{(n)} $ converges then $\sum_{n=0}^{\infty} f_n(z) = \sum_{j=0}^{\infty} A_j z^j$. } 
 \item Conclude that 
  $$ \wp'(z)^2 - 4 \wp(z)^3 + g_2 \wp(z) + g_2 = O(z^2),$$
 as $z \to 0$, which shows that $\wp'(z)^2 - 4 \wp(z)^3 + g_2 \wp(z) + g_2$ is analytic at the origin of $\CC$. 
 Here $g_2 = 60 G_4$ and $g_3 = 140 G_6$.
 \item Conclude that $\wp'(z)^2 - 4 \wp(z)^3 + g_2 \wp(z) + g_2$ is constant. (Hint: use that elliptic functions without poles are constant.)
 \item Show the constant in the previous number is zero.
\end{enumerate}

\item The zeros of $\wp(z)-c$ are simple with precisely double zeros at the points congruent to $\omega_1/2, (\omega_1+\omega_2)/2, \omega_2/2$. (Hint: what are the zeros of $\wp'(z)$ and what does this mean?)

\end{enumerate}


\end{document}
